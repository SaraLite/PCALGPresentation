\begin{frame}{Introduction}

    \begin{itemize}    
        \item There are many approaches to multiple sequence alignment:
        \begin{enumerate}
            \item Exact methods.
            \item Progressive alignment (e.g., ClustalW).
            \item Iterative approaches (e.g., PRALINE, IterAlign, MUSCLE).
            \item Consistency-based methods (e.g., MAFFT, ProbCons).
            \item Structure-based methods: include information about one or more known 3D protein structures.
        \end{enumerate}
        \item MUSCLE and MAFFT are faster $\rightarrow$ more useful for aligning large numbers of sequences. ProbCons and T-Coffee are slower $\rightarrow$ more accurate.
        \item Regarding the pairwise alignments: in progressive alignments, their scores are generated and used to build a tree; in consistency-based methods, the information about the multiple sequence alignment is used as it is being generated to guide them.
    \end{itemize}
    
\end{frame}
