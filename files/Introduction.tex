\begin{frame}{Introduction}

    \begin{itemize}
        \item Multiple sequence alignment (MSA) $\rightarrow$ way of identifying and visualizing patterns of sequence conservation. It facilitates evolutionary and phylogenetic studies. There are many approaches to multiple sequence alignment:
        \begin{enumerate}
            \item Exact methods.
            \item Progressive alignment (e.g., ClustalW).
            \item Iterative approaches (e.g., PRALINE, IterAlign, MUSCLE).
            \item Consistency-based methods (e.g., MAFFT, ProbCons).
            \item Structure-based methods: include information about one or more known 3D protein structures.
        \end{enumerate}
    \end{itemize}
    
\end{frame}

\begin{frame}{Introduction: method's approaches}

    \begin{itemize}
        \item Dynamic programming $\rightarrow$ too inefficient for more than a few sequences. Instead, heuristic strategies: tree-based progressive alignment, sequences are assembled via several pairwise alignment steps. Errors at early stages propagate and may increase the likelihood of misalignment (alleviated by post-processing steps).
        \item Consistency-based techniques $\rightarrow$ use evidence from intermediate sequences to guide the pairwise alignment (adjusting the score for a residue pairing according to support from the position of a third sequence that aligns to the others). That is, multiple sequence information is used, as it is being generated.
        \item COFFEE (another consistency-based) $\rightarrow$ a library is computed by merging consistent CLUSTALW global and LALIGN local pairwise alignments to form three-way alignments, which are assigned weights. The score for the pairwise alignment is the sum of the weights of all alignments in the library containing that aligned residue pair.
    \end{itemize}

\end{frame}